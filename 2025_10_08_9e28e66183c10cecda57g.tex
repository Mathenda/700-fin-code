\documentclass[10pt]{article}
\usepackage[utf8]{inputenc}
\usepackage[T1]{fontenc}
\usepackage{amsmath}
\usepackage{amsfonts}
\usepackage{amssymb}
\usepackage[version=4]{mhchem}
\usepackage{stmaryrd}
\usepackage{graphicx}
\usepackage[export]{adjustbox}
\graphicspath{ {./images/} }
\usepackage{caption}
\usepackage{multirow}
\usepackage{hyperref}
\hypersetup{colorlinks=true, linkcolor=blue, filecolor=magenta, urlcolor=cyan,}
\urlstyle{same}

\title{Playing the Game of Snake with Limited Knowledge: Unsupervised Neuro-Controllers Trained Using Particle Swarm Optimization }

\author{Christopher W. Cleghorn\\
Department of Computer Science University of Pretoria\\
Pretoria, Gauteng, South Africa\\
Email: ccleghorn@cs.up.ac.za}
\date{}


\begin{document}
\maketitle
\captionsetup{singlelinecheck=false}


\begin{abstract}
Methods in the domain of artificial intelligence (AI) have been applied to develop agents capable of playing a variety of games. The single-player variant of Snake is a well-known and popular video game that requires a player to navigate a line-based representation of a snake through a two-dimensional playing area, while avoiding collisions with the walls of the playing area and the body of the snake itself. A score and the snake length are increased whenever the snake is moved through items representing food. The game thus becomes more challenging as the score increases. The application of AI techniques to playing the game of Snake has not been very well explored. This paper proposes a novel technique that uses particle swarm optimization for the unsupervised training of neuro-controllers in order to play the game of Snake. The proposed technique assumes nothing about effective game playing strategies, and thus works with limited knowledge. Sensory input is also minimal. Due to the lack of similar AI-based approaches for playing Snake, the proposed technique is empirically compared against three hand-designed Snake playing agents in terms of several performance measures. The performance of the proposed technique demonstrates the feasibility of the approach, and suggests that future research into AI-based controllers for Snake will be fruitful.
\end{abstract}



\section*{B. Analysis of Results}
Tables II to IV present empirical results. The $p$-value of a Mann-Whitney $U$ test is set in bold whenever a statistically significant performance difference was identified.

Table II shows that the proposed method had better point scores than the hand-optimized agents for all playing area sizes. Larger areas showed a greater score differences.

Considering the games won (see Table III), the neurocontrollers far outperformed the simple agents on all playing area sizes except $7 \times 7$ and $9 \times 9$, where no games were won on the largest playing area. The performance difference for $8 \times 8$ areas was also quite small. The poor large board performance of the neuro-controllers is possibly due to insufficient training time. The neuro-controllers performed fairly poorly in $5 \times 5$ and $7 \times 7$ spaces, probably because no safe movement patterns exist for spaces with an odd row and column count.

Table IV shows that the neuro-controllers always produced more moves than agents $A_{2}$ and $A_{3}$. Interestingly, agent $A_{1}$ always produced more moves than the neuro-controllers, probably because it only aims to avoid collisions.

\begin{table}[h]
\begin{center}
\captionsetup{labelformat=empty}
\caption{TABLE II\\
Comparison of Mean Points Scored by the Techniques}
\begin{tabular}{|l|l|l|l|l|l|l|}
\hline
\multirow{2}{*}{Grid} & \multicolumn{2}{|c|}{PSO-NN} & \multicolumn{3}{|c|}{Simple Agents} & \multirow{2}{*}{$p$-value} \\
\hline
 & $\overline{\boldsymbol{p}}$ & $\sigma_{p}$ & Agent & $\bar{p}$ & $\sigma_{p}$ &  \\
\hline
$3 \times 3$ & 7.961 & 0.142 & \begin{tabular}{l}
$A_{1}$ \\
$A_{2}$ \\
$A_{2}$ \\
\end{tabular} & \begin{tabular}{l}
5.887 \\
6.550 \\
6.778 \\
\end{tabular} & \begin{tabular}{l}
0.190 \\
0.152 \\
0.162 \\
\end{tabular} & \begin{tabular}{l}
$7.945 \times 10^{-19}$ \\
$7.933 \times 10^{-19}$ \\
$8.437 \times 10^{-19}$ \\
\end{tabular} \\
\hline
$4 \times 4$ & 13.702 & 1.131 & \begin{tabular}{l}
$A_{1}$ \\
$A_{2}$ \\
$A_{2}$ \\
\end{tabular} & \begin{tabular}{l}
6.495 \\
8.625 \\
9.834 \\
\end{tabular} & \begin{tabular}{l}
0.257 \\
0.287 \\
0.292 \\
\end{tabular} & \begin{tabular}{l}
$7.039 \times 10^{-18}$ \\
$7.028 \times 10^{-18}$ \\
$4.190 \times 10^{-17}$ \\
\end{tabular} \\
\hline
$5 \times 5$ & 18.094 & 1.790 & \begin{tabular}{l}
$A_{1}$ \\
$A_{2}$ \\
$A_{2}$ \\
\end{tabular} & \begin{tabular}{l}
6.066 \\
10.379 \\
11.739 \\
\end{tabular} & \begin{tabular}{l}
0.254 \\
0.368 \\
0.449 \\
\end{tabular} & \begin{tabular}{l}
$7.052 \times 10^{-18}$ \\
$7.055 \times 10^{-18}$ \\
$7.053 \times 10^{-18}$ \\
\end{tabular} \\
\hline
$6 \times 6$ & 22.040 & 3.778 & \begin{tabular}{l}
$A_{1}$ \\
$A_{2}$ \\
$A_{2}$ \\
\end{tabular} & \begin{tabular}{l}
5.941 \\
11.550 \\
12.767 \\
\end{tabular} & \begin{tabular}{l}
0.192 \\
0.376 \\
0.402 \\
\end{tabular} & \begin{tabular}{l}
$7.039 \times 10^{-18}$ \\
$7.050 \times 10^{-18}$ \\
$7.034 \times 10^{-18}$ \\
\end{tabular} \\
\hline
$7 \times 7$ & 25.141 & 5.236 & \begin{tabular}{l}
$A_{1}$ \\
$A_{2}$ \\
$A_{2}$ \\
\end{tabular} & \begin{tabular}{l}
5.754 \\
12.806 \\
13.435 \\
\end{tabular} & \begin{tabular}{l}
0.236 \\
0.507 \\
0.453 \\
\end{tabular} & \begin{tabular}{l}
$7.045 \times 10^{-18}$ \\
$9.788 \times 10^{-17}$ \\
$1.346 \times 10^{-16}$ \\
\end{tabular} \\
\hline
$8 \times 8$ & 29.699 & 6.545 & \begin{tabular}{l}
$A_{1}$ \\
$A_{2}$ \\
$A_{2}$ \\
\end{tabular} & \begin{tabular}{l}
5.549 \\
14.029 \\
13.588 \\
\end{tabular} & \begin{tabular}{l}
0.206 \\
0.470 \\
0.461 \\
\end{tabular} & \begin{tabular}{l}
$7.028 \times 10^{-18}$ \\
$7.057 \times 10^{-18}$ \\
$7.053 \times 10^{-18}$ \\
\end{tabular} \\
\hline
$9 \times 9$ & 32.161 & 5.654 & \begin{tabular}{l}
$A_{1}$ \\
$A_{2}$ \\
$A_{2}$ \\
\end{tabular} & \begin{tabular}{l}
5.248 \\
14.974 \\
13.698 \\
\end{tabular} & \begin{tabular}{l}
0.209 \\
0.428 \\
0.535 \\
\end{tabular} & \begin{tabular}{l}
$7.028 \times 10^{-18}$ \\
$5.630 \times 10^{-17}$ \\
$7.954 \times 10^{-18}$ \\
\end{tabular} \\
\hline
\end{tabular}
\end{center}
\end{table}

\begin{table}[h]
\begin{center}
\captionsetup{labelformat=empty}
\caption{TABLE III\\
Comparisons of Mean Games Won by the Techniques}
\begin{tabular}{|l|l|l|l|l|l|l|}
\hline
\multirow{2}{*}{Grid} & \multicolumn{2}{|c|}{PSO-NN} & \multicolumn{3}{|c|}{Simple Agents} & \multirow{2}{*}{$p$-value} \\
\hline
 & $\bar{q}$ & $\sigma_{q}$ & Agent & $\overline{\boldsymbol{q}}$ & $\sigma_{q}$ &  \\
\hline
\multirow{3}{*}{$3 \times 3$} &  &  & $A_{1}$ & 21.460 & 5.015 & $7.763 \times 10^{-19}$ \\
\hline
 & 97.240 & 8.826 & $A_{2}$ & 30.080 & 4.597 & $7.692 \times 10^{-19}$ \\
\hline
 &  &  & $A_{2}$ & 36.160 & 5.052 & $7.749 \times 10^{-19}$ \\
\hline
\multirow{3}{*}{$4 \times 4$} &  &  & $A_{1}$ & 2.220 & 1.314 & $3.058 \times 10^{-17}$ \\
\hline
 & 64.920 & 22.700 & $A_{2}$ & 2.620 & 1.563 & $7.006 \times 10^{-17}$ \\
\hline
 &  &  & $A_{2}$ & 8.760 & 3.335 & $1.685 \times 10^{-14}$ \\
\hline
\multirow{3}{*}{$5 \times 5$} &  &  & $A_{1}$ & 0.020 & 0.141 & $5.337 \times 10^{-14}$ \\
\hline
 & 5.720 & 6.452 & $A_{2}$ & 0.000 & 0.000 & $2.127 \times 10^{-14}$ \\
\hline
 &  &  & $A_{2}$ & 0.200 & 0.404 & $4.100 \times 10^{-11}$ \\
\hline
\multirow{3}{*}{$6 \times 6$} & \multirow{3}{*}{10.620} & \multirow{3}{*}{21.305} & $A_{1}$ & 0.000 & 0.000 & $2.111 \times 10^{-15}$ \\
\hline
 &  &  & $A_{2}$ & 0.000 & 0.000 & $2.111 \times 10^{-15}$ \\
\hline
 &  &  & $A_{2}$ & 0.020 & 0.141 & $5.618 \times 10^{-15}$ \\
\hline
\multirow{3}{*}{$7 \times 7$} & \multirow{3}{*}{0.060} & \multirow{3}{*}{0.314} & $A_{1}$ & 0.000 & 0.000 & $1.594 \times 10^{-1}$ \\
\hline
 &  &  & $A_{2}$ & 0.000 & 0.000 & $1.594 \times 10^{-1}$ \\
\hline
 &  &  & $A_{2}$ & 0.000 & 0.000 & $1.594 \times 10^{-1}$ \\
\hline
\multirow{3}{*}{$8 \times 8$} & \multirow{3}{*}{1.220} & \multirow{3}{*}{4.220} & $A_{1}$ & 0.000 & 0.000 & $3.462 \times 10^{-3}$ \\
\hline
 &  &  & $A_{2}$ & 0.000 & 0.000 & $3.462 \times 10^{-3}$ \\
\hline
 &  &  & $A_{2}$ & 0.000 & 0.000 & $3.462 \times 10^{-3}$ \\
\hline
\multirow{3}{*}{$9 \times 9$} & \multirow{3}{*}{0.000} & \multirow{3}{*}{0.000} & $A_{1}$ & 0.000 & 0.000 & N/A \\
\hline
 &  &  & $A_{2}$ & 0.000 & 0.000 & N/A \\
\hline
 &  &  & $A_{2}$ & 0.000 & 0.000 & N / A \\
\hline
\end{tabular}
\end{center}
\end{table}

\section*{V. Conclusion}
This paper proposed a novel method for training neurocontrollers using PSO, in order to play Snake. Existing Snake playing methods were described, each of which are incomparable to the new method. The performance of the proposed technique was thus compared to hand-optimized agents with the same sensory input as the new method. The neurocontroller method clearly outperformed the hand-optimized agents, demonstrating the approach's feasibility.

Future work will aim to improve the proposed method, particularly for larger playing areas. Different parameters, varying stopping conditions, and performance over training will be investigated. Comparisons to additional AI-based methods will be performed, possibly using variants of the methods discussed in Section II that use the same features as the novel method.

\begin{table}[h]
\begin{center}
\captionsetup{labelformat=empty}
\caption{TABLE IV\\
Comparison of Mean Moves Made by the Techniques}
\begin{tabular}{|l|l|l|l|l|l|l|}
\hline
\multirow{2}{*}{Grid} & \multicolumn{2}{|c|}{PSO-NN} & \multicolumn{3}{|c|}{Simple Agents} & \multirow{2}{*}{$p$-value} \\
\hline
 & $\bar{r}$ & $\sigma_{r}$ & Agent & $\bar{r}$ & $\sigma_{r}$ &  \\
\hline
\multirow{3}{*}{$3 \times 3$} &  &  & $A_{1}$ & 36.067 & 1.632 & $7.050 \times 10^{-18}$ \\
\hline
 & 24.072 & 1.722 & $A_{2}$ & 17.037 & 0.570 & $7.047 \times 10^{-18}$ \\
\hline
 &  &  & $A_{2}$ & 17.409 & 0.551 & $7.951 \times 10^{-18}$ \\
\hline
\multirow{3}{*}{$4 \times 4$} &  &  & $A_{1}$ & 87.910 & 4.287 & $7.064 \times 10^{-18}$ \\
\hline
 & 58.831 & 6.749 & $A_{2}$ & 32.464 & 1.175 & $7.061 \times 10^{-18}$ \\
\hline
 &  &  & $A_{2}$ & 41.708 & 1.331 & $2.704 \times 10^{-15}$ \\
\hline
\multirow{3}{*}{$5 \times 5$} &  &  & $A_{1}$ & 155.604 & 7.987 & $1.443 \times 10^{-10}$ \\
\hline
 & 128.704 & 22.734 & $A_{2}$ & 49.805 & 2.255 & $7.066 \times 10^{-18}$ \\
\hline
 &  &  & $A_{2}$ & 76.798 & 3.190 & $9.526 \times 10^{-17}$ \\
\hline
\multirow{3}{*}{$6 \times 6$} &  &  & $A_{1}$ & 238.996 & 11.098 & $1.263 \times 10^{-7}$ \\
\hline
 & 210.344 & 47.864 & $A_{2}$ & 64.537 & 2.454 & $7.064 \times 10^{-18}$ \\
\hline
 &  &  & $A_{2}$ & 117.514 & 4.046 & $2.197 \times 10^{-17}$ \\
\hline
\multirow{3}{*}{$7 \times 7$} &  &  & $A_{1}$ & 350.627 & 16.807 & $8.317 \times 10^{-6}$ \\
\hline
 & 312.692 & 108.951 & $A_{2}$ & 80.799 & 3.788 & $1.349 \times 10^{-16}$ \\
\hline
 &  &  & $A_{2}$ & 158.839 & 5.353 & $1.312 \times 10^{-15}$ \\
\hline
\multirow{3}{*}{$8 \times 8$} &  &  & $A_{1}$ & 469.400 & 21.495 & $8.713 \times 10^{-3}$ \\
\hline
 & 443.488 & 138.502 & $A_{2}$ & 98.518 & 4.028 & $7.064 \times 10^{-18}$ \\
\hline
 &  &  & $A_{2}$ & 199.778 & 7.068 & $8.108 \times 10^{-15}$ \\
\hline
\multirow{3}{*}{$9 \times 9$} &  &  & $A_{1}$ & 598.626 & 28.272 & $1.689 \times 10^{-3}$ \\
\hline
 & 559.173 & 155.054 & $A_{2}$ & 114.007 & 3.679 & $1.349 \times 10^{-16}$ \\
\hline
 &  &  & $A_{2}$ & 239.083 & 9.915 & $1.430 \times 10^{-16}$ \\
\hline
\end{tabular}
\end{center}
\end{table}

\end{document}